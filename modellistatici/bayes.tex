\section{Stima di Bayes}
Essendo la stima di Bayes\index{Stima di Bayes} una stima a posteriori, abbiamo conoscenze a priori sui parametri $\theta$, quindi dobbiamo formulare nuove ipotesi:

  \begin{gather*}
     \exists \quad \text{un modello vero} \quad Y=\Phi\theta+V, \quad V\sim N(0,\Sigma_V) \\
     \Sigma_V>0,\quad \Theta\sim N(m_\Theta,\Sigma_\Theta),\quad \Sigma_\Theta>0,\quad  V,\Theta \quad \text{indipendenti} 
  \end{gather*}

\paragraph{Teorema} sotto le ipotesi appena citate, la stima di Bayes corrisponde a:
  \[ \theta^B=\arg \min_{\theta} \{ \varepsilon^T\Sigma_V^{-1} \varepsilon + (\theta-m_\Theta)^T\Sigma_\Theta^{-1}(\theta-m_\Theta)\} \]
che ha il suo punto di minimo in corrispondenza di:
  \[ \theta^B=(\Phi^T\Sigma_V^{-1}\Phi+\Sigma_\Theta^{-1})^{-1}(\Phi^T\Sigma_V^{-1}Y+\Sigma_\Theta^{-1}m_\Theta) \]
e la sua varianza corrisponde a:
  \[ Var[\theta^B]=(\Phi^T\Sigma_V^{-1}\Phi+\Sigma_\Theta^{-1})^{-1} \]
\paragraph{Osservazione 1} $\Sigma_\Theta^{-1} \rightarrow 0 $ allora $\theta^B \rightarrow \theta^{ML}$
\paragraph{Osservazione 2} non è necessario ipotizzare $rank[\Phi]=q$ perché sappiamo già che $\Sigma_\Theta>0$ e quindi $(\Phi^T\Sigma_V^{-1}\Phi+\Sigma_\Theta^{-1})^{-1}>0$ e quindi è invertibile
\paragraph{Osservazione 3} sfruttando informazioni a priori è possibile il caso $q>N$

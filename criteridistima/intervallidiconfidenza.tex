\section{Intervalli di confidenza}
\index{Intervallo di confidenza}
Fare delle stime senza dare alcuna indicazione sulla loro precisione può essere poco significativo e/o addirittura fuoriviante, ad esempio:

%FIXME fare i grafici
  \begin{figure}[htbp]
    \centering
    %\includegraphics[scale=0.6]{img/grafIntConf.png}
    \caption{Grafici di intervalli di confidenza \label{fig:grafIntConf}}
  \end{figure}
Come si vede dai grafici in figura \ref{fig:grafIntConf}, nel primo caso abbiamo una stima che potrebbe cadere in un intervallo molto ampio, quindi è meno affidabile della stima del secondo caso, dove l'intervallo è più stretto.\newline
Data una stima è desiderabile che questa cada all'interno di un intervallo piccolo. Con il valore $\gamma$ indichiamo il livello di confidenza della stima, ovvero la probabilità che la stima cada in un certo intervallo.


\begin{esempio} % ###########
Avendo a disposizione la stima a posteriori con ddp unimodale e simmentrica, con $\gamma=0.9$ abbiamo il 90\% di probabilità di cadere dell'intervallo $I_{0.9}=[\theta^B-\sigma \cdot 0.9 ; \theta^B + \sigma \cdot 0.9]$ \newline
[grafico]\newline
L'intervallo che abbiamo definito è detto intervallo di confidenza al 90\%
\end{esempio}

Sia dato uno stimatore $\hat{\theta}$ gaussiano non polarizzato avente varianza nota; risolviamo ora il problema di trovare il suo intervallo di confidenza $I_\gamma$ dato $\gamma$. Per risolvere il problema è utile standardizzare con una gaussiana standard e quindi calcolare:

  \[ Z=\frac{\hat{\theta}-\theta}{\sigma_{\hat{\theta}}}\sim G(0,1) \]
  
Ora, utilizzando le tabelle della gaussiana standard, troviamo:

  \[  z_0 | P(-z_o \leq Z \leq z_0)=P(\left| Z \right| \leq z_0 )=\gamma \]
  
  \[ 
    \begin{split}
      F_Z(z_0)-F_Z(-z_0) & =F_Z(z_0)-(1-F_Z(z_0))=2F_Z(z_0)-1=\gamma \Rightarrow \\
      F_Z(z_0) & =\frac{1+\gamma}{2} \Rightarrow P\left(\left| \frac{\hat{\theta}-\theta}{\sigma_{\hat{\theta}}} \right| \leq z_0 \right)=\gamma \Rightarrow \\
      P\left(\left| {\hat{\theta}-\theta}\right|\leq z_0\cdot \sigma_{\hat{\theta}} \right) & =\gamma\Rightarrow P(\hat{\theta}-z_0\cdot\sigma_{\hat{\theta}}\leq\theta\leq\hat{\theta}+z_0 \cdot\sigma_{\hat{\theta}})=\gamma \Rightarrow \\
      I_\gamma & =\left[\hat{\theta}-z_0\cdot\sigma_{\hat{\theta}} \quad,\quad\hat{\theta}+z_0\cdot\sigma_{\hat{\theta}}\right]
    \end{split}
  \]
  
Quello che abbiamo appena trovato è l'intervallo di confidenza, ovvero l'intervallo dove cade il $\gamma\%$ di probabilità; per trovarlo siamo partiti cercando un valore $z_0$ tale che la probabilità che i valori fossero compresi fra $\pm z_0$ fosse pari a $\gamma$.\newline
Secondo la scuola Bayesiana questo è un imbroglio perché $\theta$ non è una V.C. ma un numero ben definito che andiamo a calcolare in modo puntuale. L'imbroglio sta nel dire che $\theta \in I_\gamma$ perché una volta raccolti i dati $\hat{\theta}$ è un numero puro e non ha quindi alcuna variabilità, caratteristica tipica delle V.C. . In difesa del metodo vi è l'osservazione che ripetendo l'esperimento, e quindi ottenendo nuovi dati, $\hat{\theta}$ potrebbe cambiare e quindi è trattabile come una V.C.


\begin{esempio} % ###########
Siano date $N$ $X_i$ V.C. iid con $Var[X_i]=\sigma^2$, $\sigma^2$ nota ed $N$ grande. Trovare l'intervallo di confidenza $I_\gamma$ per $M_1$.\newline
Il parametro che dobbiamo cercare è $\theta=m$. Conosciamo già una stimatre della media, ovvero la media campionaria, quindi:
 
 \begin{gather*}
    \hat{\theta}=M_1 \\
    \sigma_{\hat{\theta}}=\sqrt{\frac{\sigma^2}{N}}=\frac{\sigma}{\sqrt{N}}
 \end{gather*}
 
dato che $N$ è grande, per il teorema del limite centrale, $M_1 \sim G(m,\frac{\sigma}{\sqrt{N}})$. Possiamo quindi scrivere l'intervallo di confidenza come:

  \[ I_\gamma = \left[ M_1-z_0\frac{\sigma}{\sqrt{N}} \quad ,\quad M_1+z_0\frac{\sigma}{\sqrt{N}} \right]  \]
\end{esempio}
\begin{esempio} % ###########
Siano dati $N=100$ campioni, $\gamma=0.9$ : calcolare l'intervallo di confidenza $I_\gamma$.

  \[ \frac{1+\gamma}{2}=0.95 \Longrightarrow F(z_0)=0.95 \]
  
Dalle tabelle cerchiamo ora $z_0 | F(z_0)=0.95$:

  \[ z_0=1.645 \Longrightarrow I_{0.9}=\left[ M_1-1.645\frac{\sigma}{10} \quad ,\quad M_1+1.645\frac{\sigma}{10} \right]  \]
\end{esempio}
\begin{esempio} % ###########
Siano dati $N=100$ campioni e l'intervallo di confidenza $I_\gamma=\left[ M_1-\delta \quad ,\quad M_1+\delta \right[ $: calcolare $\gamma$.\newline
Ipotizzando che $\delta=z_0\frac{\sigma}{\sqrt{N}}$ possiamo calcolare $z_0=\frac{\delta\sqrt{N}}{\sigma}$. A questo punto dalle tabelle possiamo ricavare il valore di $F(z_0)$, quindi risolvendo la seguente equazione troviamo $\gamma$:

  \[ F(z_0)=\frac{1+\gamma}{2} \Longrightarrow \gamma= 2F(z_0)-1 \]
\end{esempio}
\begin{esempio} % ###########
Sia dato l'intervallo di confidenza $I_\gamma=\left[ M_1-\delta \quad ,\quad M_1+\delta \right] $, con $\gamma=0.95$: calcolare il numero di campiono $N$. In pratica il problema si può esprimere anche nel seguente modo: quanti campioni occorre raccogliere perché si cada al 95\% dentro all'intervallo di confidenza $I_{0.95}$ ?\newline
Innanzi tutto possiamo ricavare $z_0$:

  \[ F(z_0)=\frac{1+\gamma}{2}=0.975 \Longrightarrow z_0=1.96 \]
  
Supponendo poi che $\delta=z_0\frac{\sigma}{\sqrt{N}}$, possiamo calcolare $N$ come:

  \[ \delta=z_0\frac{\sigma}{\sqrt{N}} \Longrightarrow N=z_0^2\left( \frac{\sigma}{\delta}\right) ^2 \]
  
Da notare che la precisione è molto costosa, infatti, se volessimo un $\delta$ piccolo in numero di campioni $N$ necessario cresce rapidamente:

\[ \lim_{\delta \rightarrow 0} {N}=\infty \]
\end{esempio}


Quello che abbiamo fatto fin ore è far riferimento ad una standardizzazione con una gaussiana; un'alternativa, utile per piccoli campioni, è la $t$ di Student, definita come:
\[ T_N:=\frac{Z}{\sqrt{\sum_{i=1}^{N}Z_i^2}} \cdot \sqrt{N}=\frac{Z}{\chi_N} \cdot \sqrt{N} \]
dove $Z$ e $Z_i$ sono V.C. iid $\sim G(0,1)$; $T_N$ è detta $t$ di Student a $N$ gradi di libertà, ed ha un andamento come in figrua \ref{fig:grafIntConfTstud}:
  \begin{figure}[htbp]
    \centering
    %\includegraphics[scale=0.6]{img/grafIntConf.png}
    \caption{Grafici di intervalli di confidenza \label{fig:grafIntConfTstud}}
  \end{figure}
L'andamento è simile ad una gaussiana, ma con una varianza maggiore. Per la $t$ di Student valgono le seguenti proprietà:

  \begin{gather*}
    E[T_N]\\
    Var[T_N]=\frac{N}{N-2}\\
    \lim_{N\rightarrow \infty} T_N = G(0,1)
  \end{gather*}

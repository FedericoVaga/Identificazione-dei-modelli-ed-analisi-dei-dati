\chapter{Proprietà statistiche}
% ########################################################################
% ########################################################################
\section{Proprietà}
\subsection{... del valore atteso \index{Valore atteso, proprietà}}
  \begin{align*}
    E[aX]&=aE[X]\\
    E[X+Y]&=E[X]+E[Y]\\
    E[XY]&=E[X]+E[Y] \quad \text{solo se X e Y indipendenti}
  \end{align*}
\subsection{... della varianza\index{Varianza, proprietà}}
  \begin{align*}
    Var[aX]&=a^2Var[X]\\
    Var[X+Y]&=Var[X]+Var[Y]+2Cov[X,Y] \quad \text{da notare che se X,Y indipendenti} Cov[X,Y]=0\\
  \end{align*}
\section{Chi-Quadro}
Siano $X_i$ delle V.C. gaussiane $i=1...N$ con $E[X_i]$ e $Var[X_i]=1$, tutte indipendenti. A partire da questa premessa, possiamo costruire una nuova V.C.:
  \begin{align*}
    \chi_N^2=\sum_{i=1}^{N}{X_i^2}
  \end{align*}
che prende il nome di "chi-quadro" a $N$ gradi di libertà; ne consegue, che esiste un'intera famiglia di V.C. per il chi-quadro. L'andamento del chi-quadro è mostrato nella seguente figura:\newline

\begin{center}[grafico del chi-quadro]\end{center}

Come si vede dal grafico $\lim_{N\rightarrow\infty}{\chi_N^2}=gaussiana$ , inoltre, possiamo dimostrare che:
  \begin{align*}
    E[\chi_N^2]=N
  \end{align*}
  \begin{align*}
    Var[\chi_N^2]=2N
  \end{align*}

\section{Sistemi dinamici lineari invarianti a tempo discreto}
I sistemi\footnote{oggetto di studio di altri corsi, si consiglia eventualmente di ripassare questi argomenti} che andiamo a considerare godono delle seguenti proprietà:
\begin{itemize}
  \item sono lineari\footnote{gli effetti seguono linearmente le cause}
  \item sono invarianti\footnote{le caratteristiche del processo non variano nel tempo}
  \item sono a tempo discreto\footnote{osserviamo il comportamento ad intervalli regolari}
\end{itemize}
e per questo definiti Sistemi dinamici lineari invarianti a tempo discreto\index{Sistemi dinamici lineari invarianti a tempo discreto}
\begin{figure}[htbp]\Large
  \centering
  \[
    \begin{CD}
      u(t) @>>> \framebox{$S$} @>>> y(t)
    \end{CD}
  \]
  \caption{Sistema dinamico \label{fig:sistemadinamico}}
\end{figure}

\noindent Questi sistemi obbediscono alla seguente relazione\footnote{è la convoluzione discreta fra le cause $u()$ e la funzione di trasferimento $g()$}:

\[ y(t)=\sum_{i=-\infty}^{t}{g(t-i)u(i)}=\sum_{j=0}^{\infty}{g(j)u(t-j)}  \]

Misurando la risposta impulsiva del sistema, vediamo che questa corrisponde alla funzione di trasferimento\index{Funzione di trasferimento} $y(t)=g(t)$ dato che la convoluzione ha un solo termine non nullo in corrispondenza di $u(t)=1,\quad t=0$
\subsection{Funzione di trasferimento}
La funzione di trasferimento\index{Funzione di trasferimento} è una funzione che mette in relazione un segnale in ingresso con uno in uscita.

  \begin{figure}[htbp]\Large
    \centering
    \[
      \begin{CD}
        u(t) @>>> \framebox{$g(t)$} @>>> y(t)
      \end{CD}
    \]
    \caption{Funzione di trasferimento nel tempo \label{fig:funztrasftempo}}
  \end{figure}
  
Nel corso del nostro studio definiamo la funzione di traferimento nel dominio della trasformata Z\index{Trasformata Z}:

  \[ G(z)=Z[g(t)]=\sum_{t=0}^{\infty}{g(t)z^{-t}} \]

Come visto nei capitoli precedenti, operare nel dominio della trasformata Z ci permette di calcolare molto facilmente la convoluzione, infatti risulterà essere:

  \[ Y(z)=G(z)U(z)\]


  \begin{figure}[htbp]\Large
    \centering
    \[
      \begin{CD}
        U(z) @>>> \framebox{$G(z)$} @>>> Y(z)
      \end{CD}
    \]
    \caption{Funzione di trasferimento nel dominio Z \label{fig:funztrasfz}}
  \end{figure}
  

Un generico sistema $S$ viene considerato a dimensioni finite\index{Sistema a dimensioni finite}, se la relazione che lega le cause $u(t)$ agli effetti $y(t)$ è esprimibile mediante un'equazione alle differenze:

  \[ y(t)=\sum_{i=1}^{n_a} {a_iy(t-i)}+\sum_{i=0}^{n_b} {b_iu(t-i)} \]
  
Se abbiamo a che fare con un sistema a dimensioni finite, allora la funzione di trasferimento $G(z)$ è razionale fratta\index{Funzione di trasferimento razionale fratta}, ovvero una frazione fra polinomi:

  \[ G(z)=\frac{\sum_{i=0}^{n_b} {b_iz^{-i}}}{1-\sum_{i=1}^{n_a} {a_iz^{-i}}}=\frac{N_G(z)}{D_G(z)} \]

da cui possiamo ricavare gli zeri\footnote{tutti i termini che annullano il numeratore} e i poli\footnote{tutti i termini che annullano il denominatore} rispettivamente da $N_G$ e $D_G$. D'ora in avanti considereremo solo sistemi a dimensioni finite, quindi con funzioni di trasferimento razionali fratte.

\begin{esempio}
  \begin{align*}
    y(t)&=ay(t-1)+u(t)+bu(t-1)\\
    Y(z)&=az^{-1}Y(z)+U(z)+bz^{-1}U(z) \Longrightarrow (1-az^{-1})Y(z)=(1+bz^{-1})U(z) \\
    &\Longrightarrow Y(z)=\frac{1+bz^{-1}}{1-az^{-1}}U(z)=\frac{z+b}{z-a}U(z)=G(z)U(z)
  \end{align*}
\end{esempio}

\paragraph{Stabilità} Un generico sistema $S$ è definito stabile (asintotticamente) se:
  
  \[ \lim_{t\rightarrow\infty}{g(t)}=0 \]

Questo perché gli effetti dell'impulso si smorzano nel tempo. Nel caso di sitemi a dimensioni finite, essi sono stabili se e solo se $\|a\|<1$, ovvero tutti i poli sono, in modulo, inferiori ad 1.

\begin{esempio}
  \begin{align*}
    y(t)&=ay(t-1)+u(t)\\
    Y(z)&=az^{-1}Y(z)+U(z) \Longrightarrow G(z)=\frac{z}{z-a}\\
    &\text{polo in }z=a \Longrightarrow G(z) \text{ stabile} \Leftrightarrow \|a\|<1
  \end{align*}
\end{esempio}

\paragraph{Guadagno} Il guadagno di un sistema è definito come:
  
  \[ \mu=G(1)=\sum_{t=0}^{\infty}g(t) \]
  
Supponiamo di avere in ingresso una funzione a scalino unitario $u(t)=sca(t)$ ed un sistema stabile con guadagno finito; in uscita il risultato tenerà asintotticamente al valore del guadagno:

  \[ \lim_{t\rightarrow\infty}{y(t)}=\mu \]

\paragraph{Teorema della risposta in frequenza}\index{Risposta in frequenza} Si consideri un sistema asintotticamente stabile. Sia $u(t)=A\sin(\omega t)$, allora $\lim_{t\rightarrow\infty}{y(t)-\tilde{y}(t)}=0$, ovvero il sistema si stabilizza a $\tilde{y}(t)$, dove:

  \[ \tilde{y}(t)=A|G(e^{j\omega})|\sin(\omega t + \arg[G(e^{j\omega})]) \]
  
In altre parole, un ingresso sinosuidale $u(t)$ di pulsazione $\omega$ produce, asintotticamente, un'uscita $y(t)$ con la stessa pulsazione ma ampiezza e fase dipendenti dalla funzione di trasferimento in $G(z)=G(e^{j\omega})$. $G(e^{j\omega})$ è detta risposta in frequenza. Un metodo rapido per intuire la forma del suo modulo $|G(e^{j\omega})|$ è il metodo del "tendone da circo". In pratica, in una rappresentazione 3D, ovunque siano presenti dei poli il modulo della risposta in frequenza tende all'infinito, mentre è uguale a 0 in corrispondenza degli zeri. 
%FIXME
[FIXME fare i grafici per il tendone da circo]
[FIXME è consigliabile quindi preferire le regioni non troppo vicine ai poli, altrimenti il sistema entra in risonanza e vi è una risposta troppo alta, idealmente infinita]

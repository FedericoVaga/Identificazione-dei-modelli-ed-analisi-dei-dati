\section{Sistemi lineari (con ingressi) stocastici}
Un sistema lineare con ingresso stocastico\index{Sistema lineare con ingresso stocastico}, non è altro che un sistema con in ingresso un PC stazionario (in senso lato) $u(t)$ con momenti del primo e secondo ordine:

  \begin{align*}
    m_u&=E[U(t)] &\text{non dipendente da}\quad &t\\
    \gamma_{uu}&=Cov[u(t),u(t+\tau)] &\text{dipende solo da}\quad &\tau
  \end{align*} 
  
\begin{figure}[htbp]\Large
  \centering
  \[
    \begin{CD}
      u(t) @>>> \framebox{$G(z)$} @>>> y(t)
    \end{CD}
  \]
  \caption{Sistema dinamico \label{fig:sistlinstoc}}
\end{figure}

\noindent Inoltre consideriamo funzioni di straferimento $G(z)$ con poli stabili\footnote{in modulo minori di 1}. Da queste premesse ne consegue che:

  \[ m_y=E[y(t)]=G(1)m_u=\mu m_u \]

\paragraph{Teorema} Se $u(t)$ è un PC stazionario e il sistema è stabile, allora anche $y(t)$ è un PC stazionario; se così non fosse non potremmo parlare di spettro di $y(t)$

Da qui possiamo fare alcune considerazioni sulla densità spettrale di potenza\footnote{qui rappresentate sia nel dominio della trasformata Z sia nel dominio della trasformata di Fourier}\index{Densità spettrale di potenza} del segnale d'uscita $y(t)$, possiamo scrivere:

  \begin{align*}
    \Gamma_{yy}(\omega)&=|G(e^{j\omega})|^2\Gamma_{uu}(\omega)\\
    \Phi_{yy}(z)&=G(z)G(z^{-1})\Phi_{uu}(z)
  \end{align*}
  
\noindent estendibile al caso vettoriale:

  \begin{align*}
    \Gamma_{yy}(\omega)&=G(e^{j\omega})\Gamma_{uu}(\omega)G^T(e^{-j\omega})\\
    \Phi_{yy}(z)&=G(z)\Phi_{uu}(z)G^{-1}(z^{-1})
  \end{align*}

\paragraph{Osservazione 1} In pratica $|G(e^{j\omega})|$ è un "modulatore" della densità spettrale di ingresso.
\paragraph{Osservazione 2} Dato $\Phi_{yy}(z)$, se riusciamo a trovare $G(z)$ stabile e tale che $\Phi_{yy}=G(z)G(z^{-1})$, allora possiamo simulare\footnote{ovvero ottenere un segnale $y(t)$ desiderato} $y(t)$ come l'uscita di un sistema $G(z)$ con ingresso un segnale $w(t)\sim WN(0,1)$. Questo perché la densità spettrale di potenza del rumore bianco è unitaria su tutto lo spettro:

  \[ Y(z)=G(z)W(z) \Longrightarrow  \Phi_{yy}(z)=G(z)G(z^{-1})\Phi_{ww}(z)=G(z)G(z^{-1}) \]
  
In questo caso si parla di PC stazionari a spettro razionale. Da osservare che in un processo di questo tipo, se $\bar{z}$ è un polo/zero allora sono anche poli/zeri anche l'inverso e i rispettivi coniugati: $\frac{1}{\bar{z}},\bar{z}^{*},\frac{1}{\bar{z}^{*}}$
